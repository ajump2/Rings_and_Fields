% main.tex
\documentclass[preview,border=12pt,12pt]{article}
\usepackage{amsmath}
\usepackage{amssymb}
\usepackage{maxiplot}

\begin{document}

\section{Solutions}
54. As an exercise, I was asked to show $4x^2+6x+3$ is a unit of $\mathbb{Z}_8[x]$, choosing our multiple as $ax+b$ we find
\[
\begin{maxima}
eq1:4*x^2+6*x+3,
eq2:a*x+b,
eq3:expand(eq1*eq2),
tex(eq3)
\end{maxima}
\]
which gives us the system of equations,$\begin{maxima*}
indexpower(equ,var):= hipow(equ,var),
powers(poly,var):= reverse(create_list(i,i,0,indexpower(poly,var))),
polycoeff(poly,var):=map(lambda([i],coeff(poly,x,i)),powers(poly,var)),
systemeq(eq,var,list):=map("=",polycoeff(eq,var),list),
printfunc(i):=print(concat(tex1(i),"\\\\")),
map(printfunc,systemeq(eq3,x,[0,0,0,1]))
\end{maxima*}$
\[\begin{cases}
\maximacurrent
\end{cases}\]
and we can reduce the coefficients $a,b\mod_8$ to see $b=3$ and $a=2$, which shows $\imaxima{tex(eq1)}$ is a unit of $\mathbb{Z}_8[x]$\\

23. I was also asked to show determine the units of the Gaussian integers. to accomplish this I choose $a+bi,c+di\in\mathbb{Z}[i]$ and found,
\[\begin{maxima}
f:expand((a+b*%i)*(c+d*%i)),     /* Integrating it */
tex(realpart(f)),
print("+"),
tex(imagpart(f)*i)
\end{maxima}
\]$\begin{maxima*}
ar1:[realpart(f)=0,imagpart(f)=1],
map(printfunc,ar1)\end{maxima*}$
which implies that we have the following system of equations,
$$\begin{cases}
\maximacurrent
\end{cases}$$
These can be evaluated and parameterized as,$\begin{maxima*}
l1:[realpart(f)=1,imagpart(f)=0],
sol:solve(l1,[a,b,c,d]),
for i : 1 thru length (%rnum_list) do(
sol : subst (t[i], %rnum_list[i], sol)),
map(printfunc,sol[1])
\end{maxima*}$\[\begin{cases}\maximacurrent\end{cases}\]
$\begin{maxima*}map(printfunc,sol[2])\end{maxima*}$\[\begin{cases}\maximacurrent\end{cases}\]$\begin{maxima*}map(printfunc,sol[3])\end{maxima*}$\[\begin{cases}\maximacurrent\end{cases}\]
Throwing out the first system, as $t_{1}^2+t_{2}^2\neq 1$ and so we wouldn't get an integer solution. We can determine that the units of $\mathbb{Z}[i]$ are $1,-1,i,-i$.
\end{document}
