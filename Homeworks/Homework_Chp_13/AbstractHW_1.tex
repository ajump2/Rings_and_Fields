\documentclass[12pt,a4paper]{article}
\usepackage[utf8]{inputenc}
\usepackage{amsmath}
\usepackage{amsfonts}
\usepackage{amssymb}
\usepackage{amsthm}
\usepackage{graphicx}
\usepackage[left=1cm,right=1cm,top=1cm,bottom=1cm]{geometry}
\setlength\parindent{0pt} %for no indentation
\usepackage{tikz}
\theoremstyle{plain}
\newtheorem{theorem}{Theorem}
\newtheorem{lemma}{Lemma}
\newtheorem{definition}{Definition}
\newtheorem{corollary}[theorem]{Corollary}
\newtheorem{case}{Case}
\newtheorem{subcase}{Subcase}[case]
\newtheorem{claim}{Claim}
\newtheorem{subclaim}{Subclaim}[claim]
\newtheorem{fact}{Fact}
\newtheorem{problem}{Problem}
\newtheorem{proposition}{Proposition}
\newtheorem{constr}{Construction}
\newtheorem{question}{Question}
\newtheorem{algo}{Algorithm}
\newtheorem{conjecture}{Conjecture}
%\usepackage{skull}
\renewcommand{\qedsymbol}{\begin{flushright}
\sc q.e.d.
\end{flushright}}
\newtheorem{remark}{Remark}

\newcommand*\circled[1]{\tikz[baseline=(char.base)]{
   \node[shape=circle,draw,inner sep=1pt] (char) {#1};}}
\newcommand{\contra}{\[
\Rightarrow\!\Leftarrow
\]}
\pagenumbering{gobble}
\begin{document}
Adam Jump\\
MATH 385\\
HW $\#1$\\

\textbf{Chapter 13 Problems:} $3,6,\circled{7},\circled{8},9,12,\circled{17},19,21,\circled{22},23,40,46,48,54$\\
\begin{itemize}
\item[7.] Show that the three properties listed in Exercise 6 are valid for $\mathbb{Z}_p$, where p is prime.\\

To reiterate, those properties are:

\begin{itemize}
\item[\textbf{a.}] $a^2=a$ implies $a=0$ or $a=1$.
\item[\textbf{b.}] $ab=0$ implies $a=0$ or $b=0$.
\item[\textbf{c.}] $ab=ac$ and $a\neq 0$ imply $b=c$.
\end{itemize}
\proof{$ $\newline
\begin{itemize}
\item[\textbf{(a.)}] B.W.O.C. Assume $a^2\mod_p=a$ and $a\neq 0 \text{ and } a\neq 1$\\
$\implies a^2=(p+1)a$\\
$\implies ap+a\mod_p=0+a=a$\\
$\text{however } p+1\not\in\mathbb{Z}_p$
\contra
$\therefore$ $a=0$ or $a=1$\\
\item[\textbf{(b.)}] B.W.O.C. Assume $a\neq 0$ and $b\neq 0$\\
$\implies a\cdot b = k\cdot p$, for $k\in \mathbb{N}$\\
$\implies a \text{ or } b = p\not\in\mathbb{Z}_p$
\contra
$\therefore$ $a=0$ or $b=0$\\
\item[\textbf{(c.)}] $ab=ac$, $a\neq0 \implies b=c$\\
%$a^{-1}\in\mathbb{Z}_p$\\
%$\implies a^{-1}ab=a^{-1}ac$, as $\mathbb{Z}_p$ is a field and so every non-zero element in a unit.\\
%$\implies b=c$\\
$\implies ab=ac$\\
$\implies ab-ac=0$\\
$\implies a(b-c)=0$\\
and we know that $a\neq 0$\\
$\implies b-c=0$\\
$\therefore b=c$
\end{itemize}
\qedsymbol
}
\newpage
\item[8.] Show that a ring is commutative if it has the property that $ab=ca$ implies $b=c$ when $a\neq 0$.\\
This is actually a chain of implications of the form: $$ab=ca,a\neq 0\implies b=c \implies R \text{ is commutative}$$
What we need to show is that for any arbitrary element $x\in R$, $ax=xa$.
\proof{$ $\newline
We know that
$ab=ca, a\neq 0 \implies b=c$.\\
Using $b=c$,\\
$\implies ab=ac$
however, by our assumption, $ab=ca$, this implies that,\\ $ab=ca=ac$\\
$\therefore ca=ac$, which shows $R$ is commutative.
\qedsymbol
}

\item[17.] Show that a ring that is cyclic under addition is commutative.

\proof{$ $\newline
Let $R=\langle a\rangle,\left\vert R \right\vert=n, n_1,n_2<n,\text{ and }, n_1<n_2 \text{ for } n_1,n_2\in\mathbb{Z}$\\
which means $R=\left\{i\cdot a\in R ~\vert ~ i\in\left[n\right]\right\}$,\\
$(n_1\cdot a)+(n_2\cdot a)$\\
$\implies (a+\dots+a)+(a+\dots+a)$\\
which by associativity implies,\\
$\implies (n_1-(n_1-n_2))\cdot a+(n_2-(n_2-n_1))\cdot a$
\qedsymbol
}

\item[22.] Let $R$ be a commutative ring with unity and let $U(R)$ denote the set of units of $R$. Prove that $U(R)$ is a group under the multiplication of $R$. (This group is called the \textit{group of units of $R$.})
Let $a,b\in U(R)$,\\
$a^{-1},b^{-1}\in U(R)$, by definition \textit{unit},\\
Show $a\cdot b^{-1}\in U(R)$,

\proof{$ $\newline
We know that $a,a^{-1},b,b^{-1}\in U(R)$.\\
This implies that $a\cdot b^{-1}\cdot b\cdot a^{-1}\in R$,\\
$\implies a\cdot 1 \cdot a^{-1}$,\\
$\implies a\cdot a^{-1} = 1$,\\
$\implies a\cdot b^{-1}\in U(R)$,\\
$\therefore U(R)\leq R$
\qedsymbol
}
\end{itemize}
\end{document}
